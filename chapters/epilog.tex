\chapter*{Conclusion}
\addcontentsline{toc}{chapter}{Conclusion}

In this thesis, we have analyzed, formally described, and implemented a newly developed framework and a tool for schema modeling and management based on Model-Driven Architecture (MDA) and previously developed tools \textit{XCase} and \textit{eXolutio}.

We have implemented the core framework functionality of complete modeling of schemas from an ontology with support of inheritance of entities and disjunction. We have created an easy-to-use user interface that provides all the concepts for the modeling and management of schemas in specifications and artifact generation.

Next, this thesis laid the foundations for future work in this area, which was described and analyzed in chapters \ref{chapters:future-requirements} and \ref{chapters:formal-background}, such as the use of non-interpreted entities, evolution, and inheritance of schemas or use of recommendation systems in modeling. These topics were complex to be analyzed alongside the core functionality of the tool and would require separate work, but they were necessary to be considered to implement the framework properly.

The tool is constantly used for modeling recommendations for publishing open data of public institutions and the government of the Czech Republic.

\bigskip

\begin{wrapfigure}{r}{0.25\textwidth}
    \centering
    \vspace{-\intextsep}
    \hspace*{-.75\columnsep}\qrcode[height=1in]{https://dataspecer.com/}
\end{wrapfigure}
Dataspecer is open-source and developed on GitHub. Technical documentation is part of the repository. User documentation with running instance and additional information about the project, and the link to the repository is available at the project website \url{dataspecer.com}.

\vfill