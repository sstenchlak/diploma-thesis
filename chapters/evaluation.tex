\chapter{Evaluation}
\label{chapters:evaluation}

To ensure that the framework for modeling and transformation works as desired, we have employed several automated unit tests that covered basic functionality.

We also continuously test the group of transformation generators against each other to quickly find a mistake. For that, we have developed the already mentioned CLI interface, which allows us to create artifacts from a schema and immediately validate test data against the given schema, then transform them into RDF using lifting, store them into a triplestore, use generated SPARQL query to obtain data back to RDF and execute lowering back to given format and again, compare validity against the schema and original data. This is, so far, performed only for XML as we are still working on the other formats.

\medskip

The application is also in active use to create FOSes (see the introduction chapter) for the Czech government. Our current goal, by which the development was highly affected, is to design a tool that would create them automatically, meeting all requirements, just by modeling schemas from an existing ontology that is being modeled \cite{kvremen2019improving} in the Czech republic.

Currently existing FOSes\footnote{\url{https://data.gov.cz/ofn/} (only in Czech)} (or OFNs from \textit{Otevřené formální normy} in Czech) consists of JSON schemas linked to other subschemas and an HTML technical document (similar looking to W3C recommendation documents, for example) in ReSpec\footnote{\url{https://respec.org/}} containing a description of all concepts used in a schema together with their meaning. Some FOSes also include XML schema, JSON-LD context, and an overview of the RDF structure, as the intent is to map data to RDF, and SPARQL queries. Also, examples of SPARQL queries and data in given formats may be present.

So far, these specifications were made by hand with the help of several scripts for generating structured texts, such as an overview of schema structure. This was, of course, not suitable for wider use, as large parts still needed to be created by hand.

The goal of this chapter is to evaluate the tool in a real-world application and compare the results with the existing FOSes.

\section{Register of rights and obligations}\label{sec:register-of-rights-and-obligations}

One well-defined group of FOSes is the Register of rights and obligations (RPP), which currently contains 13 specifications.

During the modeling, a few additional specifications were added that could be later reused by others. Reusing was used extensively, and it was also needed to reuse schemas from other FOSes than the RPP. This was not an issue as all schemas were designed in one instance of the application where individual groups of FOSes were separated by tags. However, in general, even this simple use case already shows that the governmental specifications are interconnected a lot and may be beneficial to have multiple instances of the tool under different institutions, each modeling its own specifications, yet still be able to reuse them.

\paragraph{Management of schemas} The evaluation also shows that there was no data specification that would have more than one schema. In this thesis, we haven't exactly specified how data specifications and data schemas shall be used to create schemas. The mentioned approach was used because each data specification has its documentation, and the modeler can choose which specifications will be reused, as reuse works on the specification level, not the schema level.

Although it may seem that the chosen approach was correct, there is still one schema for specification. In the language of the framework levels, each schema (PSM) had its own PIM.
\begin{enumerate}
    \item That means that during the evolution in the future, a user would need to evolve all schemas separately, which may cause a problem if one schema is forgotten during the process. On the other hand, evolving the whole set of schemas at once may be challenging as there would be many changes that need to be propagated, and it may be difficult to exclude changes from some schemas.
    \item Having a user interface ready for multiple schemas but always using one may be confusing for some users as they can ask a similar question as we do: "When do we need more than one schema?" The answer depends on how we decide the schemas are structured in data specifications. If exactly one schema belongs to a data specification, then we do not need more. But having all schemas under one specification is also a viable approach, especially when designing API for modules, as was introduced in the former example in the introduction chapter. Then, each module would correspond to one data specification having multiple schemas.
\end{enumerate}

Furthermore, the current approach does not support generating artifacts, specifically documentation, for a group of specifications. This is, for example, used by the current documentation of the RPP, as there is one general document that refers to other specifications.

It appears that the problem of schemas belonging to data specifications must be further analyzed as it may not be sufficient enough for larger projects, especially when designing schemas for a government that has multiple branches working more independently yet still needs to share specifications.

Possible solutions would include generalizing a concept of data specification to a project directory, where each schema belongs to one project, and the project may belong to another one (not creating a cycle). However, this would require analyzing how the PIM level shall work, whether each directory shall inherit (see \autoref{section:inheriting-schemas}) it from the parent and how the evolution shall work.

\paragraph{Schema modeling} Regarding the modeling, about half of the schemas contained maximally ten entities, and their purpose was to be referenced from others. The other half had about 20 or 30 entities at most. Most schemas used only standard constructs, such as classes, attributes, associations, and reverse associations. See \autoref{fig:screenshot} with a screenshot from the application with one of the schemas. About a third of the schemas referenced others. The references did not have cycles, but some of them had lengths of three.

Three schemas required disjunction, specifically the inheritance of classes as specified in \autoref{requirement:inheritance}, and one schema required the inheritance on the root level. Although the sample is small, it shows that in typical cases, we do not need the disjunction per se, only the inheritance, as usually, when we need to select between two different things, they are usually of the same type.

\smallskip

This specific use case of generating FOSes for the Czech government requires that if the schema represents an array, it shall be an object containing that array. Unfortunately, with the current state of development, this can not be supported as the creation of non-interpreted classes is not trivial, as it may break the generation of transformation scripts, for example, and hence need to be further analyzed. Currently, those affected schemas can be altered by hand by adding a wrapper object. In the future, this problem can be addressed on two levels.

\begin{enumerate}
    \item Either introduce a generator that does it automatically, as this is the required behavior for Czech FOSes.
    \item Or add support for non-interpreted entities and model the schema with them.
\end{enumerate}

Although the latter approach may seem cleaner yet harder for the user, it may not be correct, as adding the wrapper object may not have the desired semantics. For example, suppose that we would like to reference such schema. Should the wrapper object be present as it is a part of the schema, or do we want to reference the interpreted class instead and set the cardinality correctly?

\smallskip

Another similar requirement is that each interpreted class shall have a \textit{type} attribute with a string value that follows specific pattern rules based on the type of property. This is a very similar problem to the previous one, as the \textit{type} may be considered a domain-specific attribute that is generated automatically.

\paragraph{Documentation} Some FOSes had examples that we do not support yet, and it is a subject for future work.

Comparing the generated documentation, our results are missing some schema unrelated info, such as the specification's author or European Union logo. In general, this shall be solved by introducing a new documentation generator that works similarly but adds those missing information and sets the structure as required.

Our tool successfully generated descriptions for the conceptual model with diagrams (which some FOSes do not have) and the documentation of the schemas. The schema documentation, however, is a little harder to read. Therefore we need to focus on improving this as well.

In general, because the original documentation was hard to make by hand and various scripts were used to generate it, it shouldn't be challenging to achieve almost the same result by modifying the generator accordingly.
