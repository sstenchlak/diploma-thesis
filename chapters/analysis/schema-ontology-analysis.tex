\section{Data modeling analysis}

\subsection{Type coherency}\label{subsection:type-coherency}

As already mentioned, an ontology is not just a supporting source for the modeling process but rather the only source we can use to create schemas. The schema then represents a mapping to the ontology for further processing.

Because parts of the schema are mapped, we can check whether the attributes and associations belong to the given class. This allows us to check whether the schema is being built correctly and to provide the appropriate help during the modeling based on the type of the classes.

Although the problem may seem trivial, there are advanced scenarios that must be considered.

\begin{enumerate}
  \item We may want to add additional attributes and associations directly into the schema without a connection to the ontology. This is a schema-modeling problem as we may need, for example, to wrap several properties into an additional object (JSON) or a tag (XML) or add another property because the data we validate contains it.
  \item If $A$ is associated with $B$ and we have a schema with the class $A$ having $B$, then it may be possible to move attributes\footnote{Moving of properties to different classes will be kept as future work. Nevertheless, to cover some use cases, we employ a simpler construct of dematerialization. Association that is dematerialized is removed from the generated result, and all properties from the associated class are moved to its place.} from $A$ into $B$. Because for each $B$, we know to which $A$ it belongs, we do not lose any information during this process.
\end{enumerate}

As an example of the second case, suppose that our ontology has \textit{Goods} and their \textit{Variants}. Variants are colors, sizes, and materials for the given item. Indeed, all variants are made by one manufacturer. Therefore, it makes sense that the \textit{manufacturer} attribute would be associated with the \textit{Goods} class, whether the \textit{color} with the \textit{Variants}. This may not be a beneficiary for all data consumers. Therefore, a schema with the \textit{manufacturer} attribute moved into the \textit{Variants} class might be a better solution.

\begin{figure}[h!]\centering
  \begin{subfigure}[b]{.5\textwidth}
    \begin{Verbatim}[commandchars=\\\{\}]
\{
  "title": "Chair",
  "variants": [
    \{
      "price": 200,
      "color": "black",
      \textbf{"manufacturer": "IKEA"}
    \}
  ]
\}
    \end{Verbatim}
  \end{subfigure}%
  \begin{subfigure}[b]{.5\textwidth}
    \begin{Verbatim}[commandchars=\\\{\}]
\{
  "customerId": "12",
  {\color{gray!60}"personal-info": \{}
    "name": "John",
    "surname": "Doe"
  {\color{gray!60}\},}
  {\color{gray!60}"contact-info": \{}
    "address": ...,
    "phone": ...
  {\color{gray!60}\}}
\}
    \end{Verbatim}
    \end{subfigure}%
  \caption{Examples of data with some attributes moved. The former moves the attribute \textit{manufacturer} from the parent into the other class that has a counterpart in the ontology. The latter takes attributes such as \textit{name} and \textit{address} and wraps them with additional objects that do not correspond to the ontology.}
\end{figure}

However, this is too complex for the current state of development, but it gives us a chance to think about the problem in a more general way.

\subsection{Data modeling process}

So far, we have only described the desired structure of an ontology and a schema model, but we did not tackle the actual process of how the schema is created.

A user starts by selecting a root of the schema. Schema under the given root would then describe one entity of the given type, or a list of those entities, depending on the later configuration. Because a set of possible root classes is not limited in any way (or, we can say that the root has the most general type, hence can be specialized), the most suitable option is to let the user search for the class by its name, descriptions, or other parameters, depending on the given ontology format.

As soon as the root is placed in the schema, we get a context because the following attributes and associations to other classes depend on the class where the properties are being added. For that, we use a prompt dialog where it is possible to select those properties to be added.

Although we did not enforce that an ontology must support inheritance, most of them do. Therefore, the dialog also allows adding properties of the parent class.

\medskip

The process of adding properties can be more automated in the future. The tool can propose to automatically add all attributes and associations with nonzero cardinality, as it may be the desired behavior. We can perform this action even recursively to automatically design the whole schema just from the root and select a few options where the recursion shall stop.

In general, there might be cases with hundreds of available entities to add where only some of them are relevant to the current user. This is further supported by the fact that anyone can extend our ontology by introducing their classes associated with ours. To properly understand the relevance of different entities, a user profile from past choices needs to be built. This is an area of recommended systems, with many focusing specifically on model-driven engineering \cite{almonte2022recommender}, and we will consider them in our future work.