\section{Data transformations}

\begin{requirement}
    The application shall support generating transformations between different data conforming to supported schemas and RDF representation.
    \label{req:transformations}
\end{requirement}

Data transformations were also introduced at the beginning of this thesis. In general, \textbf{data transformations} are used to convert data (not schemas, but data that conform to given schemas) from one schema to another without changing its meaning.

One example may be to convert CSV to a JSON array of objects, where each object represents a row in the CSV. There are plenty of online tools to do this, but they do not understand the context of the data. Because both schemas were designed in the tool, we may exploit the knowledge of the mapping to the original ontology and correctly map columns from CSV to the fields in a JSON object.

In the context of this tool, transformation means both (i) transformation between different schemas under the same general schema and (ii) between different general schemas, if possible. As an example of the second case, we may have two general schemas for the same thing, where one is simpler than the other. For example, we may have the schema from \autoref{analysis/general-schema-representation} and similar with more attributes and associations, possibly with a different order of properties and labels. It is then possible to convert the data from the more complex schema to the simpler one by loosing the information. If default values are provided, or additional properties are optional, the transformation in the other direction should also be possible.

\medskip

Regarding the transformation process, there are plenty of ways to transform the data:
\begin{enumerate}
    \item Data engineers use \textbf{Python} with support for many formats using libraries. In this case, the transformation would mean a generated Python script with a predefined interface that takes data from one format and outputs in another. Depending on the use case, the script may be configurable (besides the possibility to configure the generation of transformation itself).
    \item There is \textbf{XSLT} (Extensible Stylesheet Language Transformations) language to transform between XML documents or from XML to XML-like, plain-text, or CSV documents. XSLT is an XML document that can be executed with an input document by an XSLT processor, producing the resulting document. A disadvantage is that the input document must be in XML format; hence, it cannot be used alone for bidirectional JSON and CSV transformation.
    \item There are mapping tools and languages, such as \textbf{RML} \cite{dimou2014rml} (RDF Mapping Language) designed explicitly for mapping purposes. RML maps common serialization frameworks, such as XML, CSV, and JSON, to RDF from a set of rules written in RDF. The translation mechanism is similar to the XSLT. Specifically for JSON, there is \textbf{JSON-LD} with simple rules to set mapping to RDF. Conversion tools are available in multiple programming languages.
\end{enumerate}

Although RML is a ready-to-use solution with support for all three technologies, it requires its own transformation toolchain. On the contrary, XSLT is a well-known technology among people working with XML and is widely supported. Our primary goal is to have transformations that are easy for stakeholders to use in their systems. Therefore, we will implement XSLT for XML while keeping RML for later.

Similar to the translation of a human text, there are two approaches. Either create a transformation for each pair, or have one standard format where all the others can be transformed and vice versa. The latter approach requires only one transformation for each new format added and is easier to debug, as there is a middle format. Because schemas are built from ontologies whose primary source is RDF, we will exploit this and have RDF as the middle format, which is another format to which we can transform data.

\medskip

We categorize two types of transformation, lifting and lowering. Lifting is a process of converting semi-structured data such as JSON, XML, or CSV into RDF. Lowering is the opposite process. By combining them, we can achieve a transformation between various formats. That means that even if we want to transform XML to CSV, which would be possible by a single XSLT document for simple structures, we would need to execute two transfomations.

\begin{figure}[h!]\centering
    \begin{tikzpicture}
        %Nodes
        \node[ontology] (ontology) at (0,0) {Ontology};

        \node[document,align=center] (rdf) at (3,-1.5) {RDF documents};

        \node[generalSchema,align=center] (schema1) at (-3,-1.5) {General\\schema};

        \node[schema,align=center] (xml1) at (-4,-3) {XML\\schema};
        \node[schema,align=center] (json1) at (-2,-3) {JSON\\schema};

        \node[document,align=center] (xml_document) at (1.5,-3.5) {XML\\document};
        \node[document,align=center] (json_document) at (4.5,-3.5) {JSON\\document};

        \draw[-latex] (rdf) -- node[above,anchor=south west,align=left] {conforms} (ontology);

        \draw[-latex] (ontology) -- (schema1);
        \draw[-latex] (schema1) -- (xml1);
        \draw[-latex] (schema1) -- (json1);
        \draw[-latex] (xml_document) to[bend left] node[below] {conforms} (xml1);
        \draw[-latex] (json_document) to[bend left] node[below] {conforms} (json1);

        \draw [->,line width=1pt, transform canvas={xshift=-1.25em}] (xml_document) -- (rdf);
        \draw [->,line width=1pt, transform canvas={xshift=-0.5em}] (rdf) -- (xml_document);

        \draw [->,line width=1pt, transform canvas={xshift=0.5em}] (json_document) -- (rdf);
        \draw [->,line width=1pt, transform canvas={xshift=1.25em}] (rdf) -- (json_document);

        \node[rectangle,fill=white] at (3,-2.35) {lifting and lowering};
    \end{tikzpicture}
    \caption{Example of data transformation. An XML document that conforms to XML schema may be lifted to RDF representation, which conforms to the ontology. The RDF can then be lowered to another format.}
\end{figure}