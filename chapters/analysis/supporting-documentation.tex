\begin{requirement}
    The application shall create supporting documents for the generated schemas.
\end{requirement}

The main goal of the tool is to model schemas from a given ontology. Nevertheless, to better understand the generated schema, the documentation, possibly with diagrams and examples, is very beneficial.

A structure of the documentation was already described in the introductory chapter and can be easily derived, as it only describes used concepts that are mapped from the schema.

Regarding examples, for the schema in \autoref{analysis/general-schema-representation}, Figures \ref{analysis/xml-json} and \ref{analysis/csv} could be automatically generated. This would require additional knowledge from the ontology as the application needs to understand that the title should be a buyable item (hence \textit{chair} and not \textit{sitting} for example) and the price should be reasonable to the item's actual price (because it may confuse users when the price would too low or too high).