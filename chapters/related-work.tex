\chapter{Related work}
\label{chapters:related-work}

This tool is follows a previous research that was carried out by the \textit{XML and Web Engineering Research Group} (XRG) at the Faculty of Mathematics and Physics of the Charles University  in years 2012-2015. The next section introduces the research's core concepts that will be analyzed and used in the thesis, followed by the section of similar work in the research area.

\section{Previous research}

In 2012 XRG introduced a novel approach to modelling of XML schemas \cite{necasky2012conceptual} for a particular domain ontology by integrating Model-Driven Development (specifically Model-Driven Architecture) techniques to separate a conceptual model describing the domain ontology and a structural model which described the concrete XML schema.

Model-Driven Architecture is a software engineering technique that abstracts development into three viewpoints: CIM as the \textit{Computation-independent Model}, PIM as the \textit{Platform-independent model} and PSM as the \textit{Platform-specific model}. XRG used only the last two layers of the architecture. PIM represents the domain ontology, as it is independent of the platform as XML. PSM as a platform specific then represents the given schema in their own designed grammar, that was translated into a final schema, such as XSD.

\begin{figure}[h]\centering
    \begin{tikzpicture}[
        squarednode/.style={rectangle, draw=blue!60, fill=blue!5, very thick, minimum size=5mm},
    ]
        %Nodes
        \node[squarednode] (pim) at (0,0) {PIM schema};
        \node[squarednode] (psm1) at (-2.5,-1.5) {PSM schema 1};
        \node (psmDot) at (0,-1.5) {...};
        \node[squarednode] (psmN) at (2.5,-1.5) {PSM schema N};

        \node[squarednode] (schema1) at (-2.5,-3) {XML schema 1};
        \node (psmDot) at (0,-3) {...};
        \node[squarednode] (schemaN) at (2.5,-3) {XML schema N};

        \node[squarednode,cascaded] (document1) at (-2.5,-5) {XML documents};
        \node (psmDot) at (0,-5) {...};
        \node[squarednode,cascaded] (documentN) at (2.5,-5) {XML documents};

        \node (psm_t)[text width=4cm,align=right] at (-7,0) {\textbf{Platform-specific level:}};
        \node (pim_t)[text width=4cm,align=right] at (-7,-1.5) {\textbf{Platform-independent level:}};
        \node (schema_t)[text width=4cm,align=right] at (-7,-3) {\textbf{Schema level:}};
        \node (ext_t)[text width=4cm,align=right] at (-7,-5) {\textbf{Extensional level:}};

        %Lines
        \draw[-latex] (psm1) -- (pim);
        \draw[-latex] (psmN) -- (pim);
        \draw[-latex] (schema1) -- (psm1);
        \draw[-latex] (schemaN) -- (psmN);
        \draw[-latex] (document1) -- node[fill=white]{conforms} (schema1);
        \draw[-latex] (documentN) -- node[fill=white]{conforms} (schemaN);
    \end{tikzpicture}
    \caption{Simplified version of an approach proposed by XRG in \cite{necasky2012conceptual}. Shared PIM layer with conceptual model is used by PSM, where schemas are defined in their own grammar. Grammar is then translated into schemas which conforms XML documents.}
\end{figure}

The main benefit of the approach presented is shared conceptual model between various XML schemas, as other works at that time had conceptual model for every schema. This was not practical as usually multiple schemas are applied in a sigle software. Authors have also formalized the model and have proven that their approach is correct, that means that (i) every conceptual schema models an XML schema, (ii) their translation algorithm from the internal model to schema respects introduced rules and is reversible and (iii) their normalization and optimalization algorithms produce semantically same schema.

\subsection{Evolution of schemas}

Focus of XRG was also directed to the evolution \cite{nevcasky2012evolution} of their proposed model to minimize the work of the data designer. As already stated in the introduction of the thesis, changes may be inevitable in large and complex systems and propagating even a small change from the domain ontology to all affected schemas is time consuming and error-prone.

They proposed, formalized and later implemented a solution in restricting the changes in PIM and PSM models to only atomic operations - simple changes in the model, such as \textit{creating a new class}, \textit{updating a name of association} or \textit{removing an attribute}. Those operations are not intended to be used by the user directly, but are simple enough to be formally defined and mapped to the corresponding operations in the level below. The proposed mapping is then used to propagate changes in the model to the schema level, more precisely from PSM to PIM, which is then translated to schema level.

\section{Similar work}

Google Scholar docela něco vrací na "model driven transformation cim pim data schema". Dále pak jsem našel možná relevantní - https://link.springer.com/article/10.1007/s10270-021-00905-x. K článkům byste se měl z univerzitní sítě dostat zdarma.