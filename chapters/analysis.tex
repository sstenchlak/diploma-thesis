\chapter{Requirement analysis}\label{chapters:analysis}

This chapter summarizes the expectations for the application in the form of requirements. The requirements are analyzed, and in the following chapter, the solution is proposed with a focus on a formal description of the framework.

\bigskip

As \textbf{stakeholders} we would consider data analysts, programmers, or at least people interested in the area of data modeling, since the typical use case of the application is (i) to design schemas for a large system of interconnected subsystems or modules, or (ii) to design a recommendation for publishing data. Both these use cases were described in the introduction of this chapter.

Because of the stakeholders' knowledge in the area of data modeling, we may keep the UI of the application more technical as the intent of all operations may be intuitive for them. Nevertheless, the basic functionality does not require advanced knowledge in the mentioned fields, so we propose an "expert mode." The user will be asked whether he or she feels to be an expert in the area, which would make available more advanced features of the application while keeping the UI simple for those interested in the basics of data modeling.

\bigskip

\section{General schema}

\begin{requirement}
A user shall be able to easily derive a \textbf{general schema} structure from the existing ontologies and then translate the structure into different known schema languages, such as JSON Schema, XSD and CSVW Schema and it shall be possible to add support for others easily.
\label{requirement:general-schema}
\end{requirement}

The basic idea behind this requirement was already explained in the introduction chapter. From an ontology, which specifies the relations between things from a real world, it should be possible to easily select relations and things that will describe a schema. The schema then describes a structure of data that represents those things.

\begin{figure}[h!]\centering
  \begin{tikzpicture}
      %Nodes
      \node[ontology] (ontology) at (0,0) {Ontology};

      \node[general-schema] (schema1) at (-3.5,-1.5) {General schema 1};
      \node (psmDot) at (0,-1.5) {...};
      \node[general-schema] (schemaN) at (3.5,-1.5) {General schema N};

      \node[schema,align=center] (xml1) at (-5.5,-3) {XML\\schema};
      \node[schema,align=center] (json1) at (-3.5,-3) {JSON\\schema};
      \node[schema,align=center] (csv1) at (-1.5,-3) {CSV\\schema};

      \node (psmDot) at (0,-3) {...};

      \node[schema,align=center] (xmlN) at (1.5,-3) {XML\\schema};
      \node[schema,align=center] (jsonN) at (3.5,-3) {JSON\\schema};
      \node[schema,align=center] (csvN) at (5.5,-3) {CSV\\schema};

      %Lines
      \draw[-latex] (ontology) -- (schema1);
      \draw[-latex] (ontology) -- (schemaN);

      \draw[-latex] (schema1) -- (xml1);
      \draw[-latex] (schema1) -- (json1);
      \draw[-latex] (schema1) -- (csv1);

      \draw[-latex] (schemaN) -- (xmlN);
      \draw[-latex] (schemaN) -- (jsonN);
      \draw[-latex] (schemaN) -- (csvN);
  \end{tikzpicture}
  \caption{Diagram showing the core workflow behind the data modelling from an ontology. User can create general schemas (blue rectangles) from the ontology from which are created traditional data schemas, such as XSD, CSV Schema or JSON schema.}
\end{figure}

\smallskip

We aim to design a model for a general schema that can describe most of the serialization data formats. This model will be used as a mapping from the ontology to the desired schema. The model must be robust enough to support different formats, as we want to use the same for all of them.

There are many formats for data exchange, the most famous being JSON, XML, CSV/TSV, and RDF. The formats can be categorized into the following categories based on the model of the structure:
\begin{itemize}
    \item \textbf{Hiearchical model} stores data in a tree-like structure, having one root thing with properties that may recursively contain other things. It was one of the most common models for data serialization in the past few decades as it was easy to understand and interpret. XML and JSON are examples of formats that use this model.
    \item \textbf{Relational model} uses a set of tables to store data. Each table represents a sequence of similar things, each on one row with columns as properties. Rows may point to rows in other tables to link data. The relational model is also famous for its simplicity in CSV and TSV files which can be easily parsed.
    \item \textbf{Graph model} represents data in general graph structure with nodes and edges. RDF (Resource Description Framework) became a popular format using the graph model, where nodes usually represent things or literal values and edges connect them as properties.
\end{itemize}

As our primary intent is to support JSON and XML, we will use the first type of model to represent data in our general format. The translation from that format to individual schemas in the hiearchical model would be implicit.

Supporting translation from the general schema, which is in the hierarchical model, to the formats in the relational and graph models should be possible in a limited way\footnote{That means we may not be able to reverse translation from specific schema to the general schema or it may not be possible to use the full power of the given specific schema. However, this is not important to us as our target is support for basic use-cases.}, which is sufficient and follows the requirement to have one general schema.

The graph model is not even necessary to generate as we use the ontology that is already in the graph model; hence we can use the ontology directly as the schema to validate our data.


%==============================================================================%
\subsection{Analysis of the formats}

We will analyze the standard formats to properly design a user interface for the schema modeling and the underlying general schema model capable of describing those formats.

\smallskip

\textbf{JSON (JavaScript Object Notation)} is a simple format with two complex data types: objects and arrays. The objects represent data in key-value pairs, with values that can have any type, including other objects and arrays. Arrays then represent lists, and both arrays as objects may be in the root of the document tree. Semantically, objects represent things, with their values as properties.

\textbf{XML (Extensible Markup Language)} is similar to JSON as both formats are hierarchical. XML tags wrap parts of the document representing either things or properties of things and can be nested similarly to the JSON format. In contrast to JSON, XML tags can have attributes.

\begin{figure}[h!]\centering
    \begin{subfigure}[b]{.45\textwidth}
\begin{Verbatim}[commandchars=\\\{\}]
\{
  "id": 3758,
  "title": "Chair",
  "variants": [
    \{
      "title": "Black",
      "price": 200,
      "color": "black"
    \},
    \{
      "title": "White",
      "price": 200,
      "color": "white"
    \}
  ]
\}
\end{Verbatim}
        \caption{JSON document - braces {\tt\{\}} wraps object and brackets {\tt[]} wraps array}
      \end{subfigure}\hfil%
      \begin{subfigure}[b]{.45\textwidth}
\begin{Verbatim}[commandchars=\\\{\}]
<Good id="3758">
  <title>Chair</title>
  <Variant>
    <title>Black</title>
    <price>200</price>
    <color>black</color>
  </Variant>
  <Variant>
    <title>White</title>
    <price>200</price>
    <color>white</color>
  </Variant>
</Good>
\end{Verbatim}

\vfill

        \caption{XML document - {\tt<Good>} tag serves as a class wrapper, whether {\tt<title>} has a property meaning}
      \end{subfigure}
    \caption{Comparison of JSON and XML format both showing data about the same chair.}
    \label{analysis/xml-json}
\end{figure}

As seen from \autoref{analysis/xml-json}, the XML format is more complex, as it supports tag attributes (see the \verb|id="3758"| attribute), and arrays can be written in two distinct ways. We can place items of the array directly in the parent container, as we can see with the \verb|<Variant>| tag, or we can wrap them into another container for clarity (for example, into \verb|<variants>| tag).

\smallskip

JSON Schema is a JSON document describing the data structure we can expect in JSON documents. For this part of the thesis, it is sufficient to know that the schema defines which root object we can expect and a set of allowed properties and their types for each object.

Suppose we have chosen a structure very similar to JSON Schema to be our general structure format. We are interested only in how it describes the document's structure, not its representation. Because JSON is simpler than XML, we can use our model to describe only simple XML documents as we are missing constructs that would describe advanced XML features.

For example,
object property {\tt x} with primitive value {\tt y} would represent an XML tag {\tt <x>y</x>}; if {\tt y} is an object, we will recursivelly apply this rule.
Object property {\tt x} with an array of {\tt y\textsubscript{i}}  would represent multiple XML tags {\tt<x>y\textsubscript{1}</x><x>\nobreak\hfil\penalty0 \hfilneg y\textsubscript{2}</x>...<x>y\textsubscript{n}</x>}.
Finally, we will start with the root tag, which was {\tt <Good>} in our case.

To describe and distinguish between more advanced XML features, we would need to add XML-specific options to our model, such as:
\begin{enumerate}
  \item For every object property with a primitive value, there should be an option that the given property become an attribute of the parent tag. For example, the \verb|id| property of the chair may be either the attribute \verb|id="3758"| of the parent, or the full tag \verb|<id>3758</id>| inside the parent.
  \item For every array property, there should be an option that the given list of tags will be wrapped.
  \item XML, compared to JSON, recognizes an order of the elements in the document. This means that we may decide whether we want to enforce the specific order or not, which can also be fixed by another option in the parent.
\end{enumerate}

Comparing structure of JSON and XML once again, we can let a user use the JSON Schema-like structure with optional annotations for advanced XML features. This allows us to have a simple model which is easy to understand and use and can be annotated by other options for specific languages, as we have shown for XML.

\smallskip

\textbf{CSV (Comma-Separated Values)} or TSV stores data in tables. This, unfortunately, means that the structure is completely different than in the case of JSON and XML. Because having a separate schema would cause complications against other requirements, we will analyze whether it is possible to translate our general structure format from a hierarchical model to a relational one.

In the general case, there are existing approaches \cite{10.1145/304181.304220, 10.1007/3-540-45271-0_10} to map hierarchical model to relational. Therefore, we will show only a brief example. Suppose our general structure format contains objects, properties, and arrays. From each object type, we will create a table with columns as properties. Each table must have a primary key so that the tables can be linked together. If the schema contains an array, we will link children to the parent table; thus, array properties will not have a column.

\begin{figure}[h!]\centering
  \begin{subfigure}{.5\textwidth}
    \centering
    \begin{tabular}{ll}\toprule
      id   & title \\ \midrule
      3758 & Chair \\ \bottomrule
    \end{tabular}%
  \end{subfigure}%
  \begin{subfigure}{.5\textwidth}
    \centering
    \begin{tabular}{llll}\toprule
      good-id & title & price & color \\ \midrule
      3758 & Black & 200 & black \\
      3758 & White & 200 & white \\ \bottomrule
    \end{tabular}
  \end{subfigure}
  \caption{Document of two CSV tables representing the same data as in the \autoref{analysis/xml-json}. Left table contains the root.}
  \label{analysis/csv}
\end{figure}

Because all the tables represent arrays, we can not formally convert the schema with an object in the root. We have suppressed this in the \autoref{analysis/csv} simply by wrapping the schema root into the array.

To support CSV documents containing unrelated data, specifically CSV tables, that do not reference each other, we may need to have a schema with multiple roots. Multiple root schemas may be helpful in some advanced data-modeling problems. We will keep the question behind this open as there are not enough use-cases right now.

Although we have not dealt with advanced cases, the model is robust enough for most use cases.

\subsection{Designing the model}

So far, we have shown that a JSON Schema-like model with format-specific annotations is sufficient for describing a structure of JSON, XML, and CSV documents. In general, we cannot have too strict requirements on the model as some other formats may not require all the information or may be too simple. This pushes us to define the schema in the most elementary way.

\medskip

We will allow only classes to be a root of schemas and instead add an option that the root can be an array. This simplifies the work with the model as we may always expect a class.

Classes then have an ordered list of properties. This is different from JSON, where properties have no order. A property may be an attribute or association. An attribute has a primitive type, such as a string or a number. Association is a property with another class. Because we have forbidden the use of arrays in the root, we omit them entirely as an array of primitive values and classes can be achieved by the cardinality of attributes and associations, respectively. Cardinality is an interval specifying how many values a property can have. $1..1$ is for required properties, $0..1$ for optional, and $0..*$ for arrays.

\smallskip

We can use two different approaches to visualize the model's hierarchical structure. Previous tools \textit{xCase} and \textit{eXolutio} used graph visualization, where nodes were used to show classes and edges to show associations. An alternative approach is to use a textual "bullet list" representation, as the model is \textit{usually} a tree.

The latter approach is easier to understand as the final product is a schema for documents that has a similar "structure" as the representation. It is easier to implement, more compact in size on the screen, and easier to work with on smaller devices. Also, the order of the properties is more intuitive, and we can use more styling options for advanced constructs.\footnote{So far, we have described only a basic schema structure. See other requirements for advanced constructs.} However, in the general case, users may benefit from the graph view if the schema refers to another schema (see the \autoref{analysis/requirement/schema-reference}) multiple times because this can be easily denoted in the graphical interface (see \autoref{analysis/difference-between-graphical-and-hiearchical}).

\begin{figure}[h!]\centering
  \begin{subfigure}{.5\textwidth}
      \centering
      \begin{tikzpicture}[
          squarednode/.style={rectangle, draw=blue!60, fill=blue!5, very thick, minimum size=5mm},
      ]
          %Nodes
          \node[ontology] (root) at (0,0) {root class};

          \node[squarednode] (a1) at (-1.5,-1.5) {association 1};
          \node[squarednode] (a2) at (1.5,-1.5) {association 2};

          \node[ontology] (ref) at (0,-3) {referenced class};

          %Lines
          \draw[-latex] (root) -- (a1);
          \draw[-latex] (root) -- (a2);
          \draw[-latex] (a1) -- (ref);
          \draw[-latex] (a2) -- (ref);
      \end{tikzpicture}
      \caption{Graphical representation}
    \end{subfigure}%
    \begin{subfigure}{.5\textwidth}
\begin{Verbatim}[commandchars=\\\{\}]
{\color{purple!60}root class}
  {\color{blue!60}- association 1} to
      {\color{purple!60}referenced class}
  {\color{blue!60}- association 2} to
      {\color{purple!60}referenced class}
\end{Verbatim}
      \caption{Hiearchical representation}
    \end{subfigure}

  \caption{Figure showing a schema referencing the same subschema twice, essencially creating a cycle in unoriented graph. Two different representations are shown - graph and hiearchical.  The former one shows that both associations refer the same subschema, which later representation can not show.}
  \label{analysis/difference-between-graphical-and-hiearchical}
\end{figure}

Because the main use-case is to generate simple or moderately advanced schemas, the textual approach is preferred. Nevertheless, the graph view might be implemented in the future.

\medskip

As shown in the \autoref{analysis/difference-between-graphical-and-hiearchical}, the schema may be represented as a "bullet list" where each class, association, or attribute is on a separate line. Classes have a list of properties under the class name. Associations point directly to other classes, and, therefore, they can be merged with the class name on a single line. Other attributes, including format-specific, will be on the line next to the item name.

\begin{figure}[h!]\centering
  \begin{Verbatim}[commandchars=\\\{\}]
{\color{purple!60}class \textbf{Good}}
  {\color{blue!60}- attribute \textbf{id}}[1..1]: string
  {\color{blue!60}- attribute \textbf{title}}[1..1]: string
  {\color{purple!60}- association \textbf{variants}}[0..*]: \textbf{Variant}
    {\color{blue!60}- attribute \textbf{title}}[1..1]: string
    {\color{blue!60}- attribute \textbf{price}}[1..1]: number
    {\color{blue!60}- attribute \textbf{color}}[1..1]: string
\end{Verbatim}
  \caption{Proposition for how the general schema may be represented for the example that validates data with the chair.}
  \label{analysis/general-schema-representation}
\end{figure}

It shall be possible to change the order of the properties by dragging them, and options for given items shall be available next to them. Attributes and associations shall be distinguished both by color and supporting graphics. More advanced constructs may have unique styling options to provide more information if necessary.

\section{Ontology}

\begin{requirement}
    \label{requirement:ontologies-on-the-web}
    As many ontologies are located on the web in formats such as OWL (Web Ontology Language), RDFs (RDF Schema), UFO (Unified Foundational Ontology), etc., the application shall support reading them.
\end{requirement}

It may seem that designing the ontology directly in the tool is beneficial because a user does not need to use other tools and the application may already build the schema, which is a feedback to the user. This approach was used in tools \textit{XCase} and \textit{eXolutio} as can be seen in the \autoref{fig:exolutio} in the left panel. However, it has the following drawbacks:

\begin{enumerate}
    \item Designing an ontology is a well-defined problem. There are many great and time-proven tools we could not cope with.
    \item Even if the ontology will be used just to generate the schemas, it may be worthy of publishing it anyways as others may benefit from it.
    \item It is better to split a complex problem into smaller ones.
\end{enumerate}

On the other hand, not having direct access to the ontology, as it will be on the Web, has the following impacts:

\begin{enumerate}
    \item The ontology may \textbf{not always be available}. Unavailability should not prevent us from generating the schemas and making minor changes to them if those changes are not directly related to exploring the ontology.
    \item The concepts in the ontology may \textbf{point to another ontology} according to the Linked Data principles.
\end{enumerate}

For the reasons mentioned above, the preferred workflow is to design the ontology separately in the external tool, publish it on the Web, and then model the schema in the application. There is \autoref{requirement:pim-editing} later in the text specifying that a user can make modifications in the application. This is not inconsistent with the statements, as it deals with minor changes instead of defining a complete ontology.

The term ontology has already been defined in the introductory chapter, and we will formally define its specific requirements in the next chapter.

It shall be easy to implement support for other types of ontologies, and all of them shall be linkable according to the LD principles.

\subsection{Format of the ontology}

In the above requirement, several different formats were proposed for the ontology. This section will analyze the minimal requirements for any ontology format and how we will treat additional information in them. Because the core goal is to design schemas, we will start with a model proposed from \autoref{requirement:general-schema}. The schema consists of classes and their properties. A class corresponds to a thing from real life. An attribute is a literal that belongs to the given class only. On the other hand, an association is a link between two (not necessarily different) classes. From this point of view, the association is an independent entity.

Associations are usually oriented, and some ontologies may specify a title and a description for a reverse direction. For example, in RDFS, the association (or property in the RDFS terminology) is an entity of type {\tt rdf:Property} having domain and range classes and a title and description. Therefore, it only describes the forward direction. We can, of course, create a property in the other direction as well, but there would not be a connection between a forward and a reverse direction. Hence, we would not know that those are semantically the same properties. UFO (Unified Foundational Ontology), as an example of a more complex ontology, introduces relators. Relators are relationships between two or more things connected by mediation. The mediation can be described, giving us a way to describe both directions differently.

Although the latter approach is more complex, using simple concepts for associations may be disadvantageous for the aforementioned reason. Therefore, we will follow the pattern of UFO and any other simpler ontologies, such as RDFS, will not have the reverse direction described.

\medskip

An ontology in the context of this requirement is a set of classes that have attributes. The associations then connect two classes together. The connected classes may not have the same ontology.

As some formats specify the ontology in a more complex way, the application may use the additional information to better design the schema. This statement is better defined in the next chapter.

\begin{figure}[h!]\centering
  \centering
  \begin{tikzpicture}[
    attribute/.style={align=center},
  ]
    \node[ontologyClass] (good) at (0,0) {<<class>>\\\textbf{Good}};
    \node[ontologyClass] (variant) at (9,0) {<<class>>\\\textbf{Good variant}};

    \node[attribute] (a11) at (-.75,1.5) {<<attr>>\\id};
    \node[attribute] (a12) at (.75,1.5) {<<attr>>\\title};

    \node[attribute] (a21) at (7.5,1.5) {<<attr>>\\title};
    \node[attribute] (a22) at (9,1.5) {<<attr>>\\price};
    \node[attribute] (a23) at (10.5,1.5) {<<attr>>\\color};

    \draw (good) -- (a11);
    \draw (good) -- (a12);

    \draw (variant) -- (a21);
    \draw (variant) -- (a22);
    \draw (variant) -- (a23);

    \draw (good) -- node[pos=0.225,below]{$\blacktriangleleft$ is variant of} node[above]{<<association>>} node[pos=0.775,below]{has variant $\blacktriangleright$} (variant);
  \end{tikzpicture}

  \caption{Schematic diagram of an ontology which could be used for the schema from the \autoref{analysis/general-schema-representation}.}
  \label{figure:example-of-ontology}
\end{figure}

% todo popis, jak by mela ontologie vypadat na zaklade toho schematu. Jakou by mela mit strukturu a co musi obsahovat

% mozna ze to je jak uml

\section{Data modeling analysis}

\subsection{Type coherency}\label{subsection:type-coherency}

As already mentioned, an ontology is not just a supporting source for the modeling process but rather the only source we can use to create schemas. The schema then represents a mapping to the ontology for further processing.

Because parts of the schema are mapped, we can check whether the attributes and associations belong to the given class. This allows us to check whether the schema is being built correctly and to provide the appropriate help during the modeling based on the type of the classes.

Although the problem may seem trivial, there are advanced scenarios that must be considered.

\begin{enumerate}
  \item We may want to add additional attributes and associations directly into the schema without a connection to the ontology. This is a schema-modeling problem as we may need, for example, to wrap several properties into an additional object (JSON) or a tag (XML) or add another property because the data we validate contains it.
  \item If $A$ is associated with $B$ and we have a schema with the class $A$ having $B$, then it may be possible to move attributes\footnote{Moving of properties to different classes will be kept as future work. Nevertheless, to cover some use cases, we employ a simpler construct of dematerialization. Association that is dematerialized is removed from the generated result and all properties from the associated class are moved on its place.} from $A$ into $B$. Because for each $B$, we know to which $A$ it belongs, we do not lose any information during this process.
\end{enumerate}

As an example of the second case, suppose that our ontology has \textit{Goods} and their \textit{Variants}. Variants are colors, sizes, and materials for the given item. Indeed, all variants are made by one manufacturer. Therefore, it makes sense that the \textit{manufacturer} attribute would be associated with the \textit{Goods} class, whether the \textit{color} with the \textit{Variants}. This may not be a beneficiary for all data consumers. Therefore, a schema with the \textit{manufacturer} attribute moved into the \textit{Variants} class might be a better solution.

\begin{figure}[h!]\centering
  \begin{subfigure}[b]{.5\textwidth}
    \begin{Verbatim}[commandchars=\\\{\}]
\{
  "title": "Chair",
  "variants": [
    \{
      "price": 200,
      "color": "black",
      \textbf{"manufacturer": "IKEA"}
    \}
  ]
\}
    \end{Verbatim}
  \end{subfigure}%
  \begin{subfigure}[b]{.5\textwidth}
    \begin{Verbatim}[commandchars=\\\{\}]
\{
  "customerId": "12",
  {\color{gray!60}"personal-info": \{}
    "name": "John",
    "surname": "Doe"
  {\color{gray!60}\},}
  {\color{gray!60}"contact-info": \{}
    "address": ...,
    "phone": ...
  {\color{gray!60}\}}
\}
    \end{Verbatim}
    \end{subfigure}%
  \caption{Examples of data with some attributes moved. The former moves the attribute \textit{manufacturer} from the parent into the other class that has a counterpart in the ontology. The latter takes attributes such as \textit{name} and \textit{address} and wraps them with additional objects that do not correspond to the ontology.}
\end{figure}

However, this is too complex for the current state of development, but it gives us a chance to think about the problem in a more general way.

\subsection{Data modeling process}

So far, we have only described the desired structure of an ontology and a schema model, but we did not tackle the actual process of how the schema is created.

A user starts by selecting a root of the schema. Schema under the given root would then describe one entity of the given type, or a list of those entities, depending on the later configuration. Because a set of possible root classes is not limited in any way (or, we can say that the root has the most general type, hence can be specialized), the most suitable option is to let the user search for the class by its name, descriptions, or other parameters, depending on the given ontology format.

As soon as the root is placed in the schema, we get a context because the following attributes and associations to other classes depend on the class where the properties are being added. For that, we use a prompt dialog where it is possible to select those properties to be added.

Although we did not enforce that an ontology must support inheritance, most of them do. Therefore, the dialog also allows adding properties of the parent class.

\medskip

The process of adding properties can be more automated in the future. The tool can propose to automatically add all attributes and associations with nonzero cardinality, as it may be the desired behavior. We can perform this action even recursively to automatically design the whole schema just from the root and select a few options where the recursion shall stop.

In general, there might be cases with hundreds of available entities to add where only some of them are relevant to the current user. This is further supported by the fact that anyone can extend our ontology by introducing their classes associated with ours. To properly understand the relevance of different entities, a user profile from past choices needs to be built. This is an area of recommended systems, with many focusing specifically on model-driven engineering \cite{almonte2022recommender}, and we will consider them in our future work. % not a requirement

\bigskip

\begin{requirement}
    The application shall create supporting documents for the generated schemas.
\end{requirement}

The main goal of the tool is to model schemas from a given ontology. Nevertheless, to better understand the generated schema, the documentation, possibly with diagrams and examples, is very beneficial.

A structure of the documentation was already described in the introductory chapter and can be easily derived, as it only describes used concepts that are mapped from the schema.

Regarding examples, for the schema in \autoref{analysis/general-schema-representation}, Figures \ref{analysis/xml-json} and \ref{analysis/csv} could be automatically generated. This would require additional knowledge from the ontology as the application needs to understand that the title should be a buyable item (hence \textit{chair} and not \textit{sitting} for example) and the price should be reasonable to the item's actual price (because it may confuse users when the price would too low or too high).

\section{Data transformations}

\begin{requirement}
    The application shall support generating transformations between different data conforming to supported schemas and RDF representation.
    \label{req:transformations}
\end{requirement}

Data transformations were also introduced at the beginning of this thesis. In general, \textbf{data transformations} are used to convert data (not schemas, but data that conform to given schemas) from one schema to another without changing its meaning.

One example may be to convert CSV to a JSON array of objects, where each object represents a row in the CSV. There are plenty of online tools to do this, but they do not understand the context of the data. Because both schemas were designed in the tool, we may exploit the knowledge of the mapping to the original ontology and correctly map columns from CSV to the fields in a JSON object.

In the context of this tool, transformation means both (i) transformation between different schemas under the same general schema and (ii) between different general schemas, if possible. As an example of the second case, we may have two general schemas for the same thing, where one is simpler than the other. For example, we may have the schema from \autoref{analysis/general-schema-representation} and similar with more attributes and associations, possibly with a different order of properties and labels. It is then possible to convert the data from the more complex schema to the simpler one by loosing the information. If default values are provided, or additional properties are optional, the transformation in the other direction should also be possible.

\medskip

Regarding the transformation process, there are plenty of ways to transform the data:
\begin{enumerate}
    \item Data engineers use \textbf{Python} with support for many formats using libraries. In this case, the transformation would mean a generated Python script with a predefined interface that takes data from one format and outputs in another. Depending on the use case, the script may be configurable (besides the possibility to configure the generation of transformation itself).
    \item There is \textbf{XSLT} (Extensible Stylesheet Language Transformations) language to transform between XML documents or from XML to XML-like, plain-text, or CSV documents. XSLT is an XML document that can be executed with an input document by an XSLT processor, producing the resulting document. A disadvantage is that the input document must be in XML format; hence, it cannot be used alone for bidirectional JSON and CSV transformation.
    \item There are mapping tools and languages, such as \textbf{RML} \cite{dimou2014rml} (RDF Mapping Language) designed explicitly for mapping purposes. RML maps common serialization frameworks, such as XML, CSV, and JSON, to RDF from a set of rules written in RDF. The translation mechanism is similar to the XSLT. Specifically for JSON, there is \textbf{JSON-LD} with simple rules to set mapping to RDF. Conversion tools are available in multiple programming languages.
\end{enumerate}

Although RML is a ready-to-use solution with support for all three technologies, it requires its own transformation toolchain. On the contrary, XSLT is a well-known technology among people working with XML and is widely supported. Our primary goal is to have transformations that are easy for stakeholders to use in their systems. Therefore, we will implement XSLT for XML while keeping RML for later.

Similar to the translation of a human text, there are two approaches. Either create a transformation for each pair, or have one standard format where all the others can be transformed and vice versa. The latter approach requires only one transformation for each new format added and is easier to debug, as there is a middle format. Because schemas are built from ontologies whose primary source is RDF, we will exploit this and have RDF as the middle format, which is another format to which we can transform data.

\medskip

We categorize two types of transformation, lifting and lowering. Lifting is a process of converting semi-structured data such as JSON, XML, or CSV into RDF. Lowering is the opposite process. By combining them, we can achieve a transformation between various formats. That means that even if we want to transform XML to CSV, which would be possible by a single XSLT document for simple structures, we would need to execute two transfomations.

\begin{figure}[h!]\centering
    \begin{tikzpicture}
        %Nodes
        \node[ontology] (ontology) at (0,0) {Ontology};

        \node[artefact,align=center] (rdf) at (3,-1.5) {RDF documents};

        \node[general-schema,align=center] (schema1) at (-3,-1.5) {General\\schema};

        \node[schema,align=center] (xml1) at (-4,-3) {XML\\schema};
        \node[schema,align=center] (json1) at (-2,-3) {JSON\\schema};

        \node[artefact,align=center] (xml_document) at (1.5,-3.5) {XML\\document};
        \node[artefact,align=center] (json_document) at (4.5,-3.5) {JSON\\document};

        \draw[-latex] (rdf) -- node[above,anchor=south west,align=left] {conforms} (ontology);

        \draw[-latex] (ontology) -- (schema1);
        \draw[-latex] (schema1) -- (xml1);
        \draw[-latex] (schema1) -- (json1);
        \draw[-latex] (xml_document) to[bend left] node[below] {conforms} (xml1);
        \draw[-latex] (json_document) to[bend left] node[below] {conforms} (json1);

        \draw [->,line width=1pt, transform canvas={xshift=-1.25em}] (xml_document) -- (rdf);
        \draw [->,line width=1pt, transform canvas={xshift=-0.5em}] (rdf) -- (xml_document);

        \draw [->,line width=1pt, transform canvas={xshift=0.5em}] (json_document) -- (rdf);
        \draw [->,line width=1pt, transform canvas={xshift=1.25em}] (rdf) -- (json_document);

        \node[rectangle,fill=white] at (3,-2.35) {lifting and lowering};
    \end{tikzpicture}
    \caption{Example of data transformation. An XML document that conforms to XML schema may be lifted to RDF representation, which conforms to the ontology. The RDF can then be lowered to another format.}
\end{figure}

\section{Data specification}

The following requirements force us to group similar general schemas into a project that we call a \textbf{data specification}. Schemas in the data specification may share some configuration or depend on each other, as we specify later. Each schema belongs to exactly one data specification. We will not determine in this thesis which schemas should share a data specification and which should not because there are currently no limitations that would state otherwise. However, this may change in the future as new requirements arise.

\begin{requirement}
  It shall be possible to refer to other schemas to use them as building blocks for larger ones. Schema reference shall be treated as a reference to the resulting schemas and documentation as well.
  \label{analysis/requirement/schema-reference}
\end{requirement}

Referencing other schemas is crucial for advanced use cases where it is essential to split large schemas into smaller blocks that can be published and used separately.

For most schema languages, it should be sufficient to refer to the other schema as is. For example, in JSON, we can use the \verb|$ref| keyword with a path to the referenced schema. On the other hand, data transformations may not always be able to handle this approach. Hence, having a full copy of the schema might be necessary. Referencing a schema would thus require access to all data in its specification.

To avoid problems with tracking references and knowing which data specification needs to be loaded to generate artifacts properly, a user would need to explicitly set a given \textbf{data specification is being reused}. Similarly to the requirement with the ontology, we do not require that the reused data specification be always available.\footnote{See the requirement X for context.} The application shall work even if the data specification is not available at the moment if the presence of the specification is not required directly, such as for creating a new reference or generating artifacts that depend on it.

\begin{figure}[h!]\centering
  \begin{tikzpicture}
      \node[dataSpecification,align=center] (ds1) at (-3,0) {Data specification 1};
      \node[dataSpecification,align=center] (ds2) at (3,0) {Data specification 2};

      \node[generalSchema,align=center] (s11) at (-4.5,-1.5) {General\\schema};
      \node[generalSchema,align=center] (s12) at (-1.5,-1.5) {General\\schema};
      \node[generalSchema,align=center] (s21) at (3,-1.5) {General\\schema};

      \draw[-latex] (ds1) -- (s11);
      \draw[-latex] (ds1) -- (s12);
      \draw[-latex] (ds2) -- (s21);

      \draw[-latex,densely dotted] (ds1) -- node[above] {reuses} (ds2);
      \draw[-latex,densely dotted] (s12) -- node[above] {refers} (s21);
  \end{tikzpicture}
  \caption{Example of reusing of specifications. All schemas from reused specification become available to refer from local schemas. Only the root of the schema may be refered.}
\end{figure}

\begin{requirement}
  The list of supported schemas, transformations, documents, and other files generated from the general schema shall be easily expandable to adapt the tool to different use cases.
\end{requirement}

The generation of schemas is robust enough to be used in every common scenario. Therefore for most users, we do not expect they may want to intervene in the process besides the standard configuration, such as indentation, using of comments, or a default language.

On the other hand, documentation is a very vague concept that neither we have adequately specified. Sometimes simple Markdown documentation may be sufficient, while elsewhere, the user may require a strict format of multiple documents in HTML.

Transformations have similar issues. There are multiple ways and technologies that transform data between different schemas. We have already mentioned transformation through the RDF format, either by RML or custom scripts, such as XSLT for XML. For the more demanding user, it is even possible to create transformation scripts between pairs of different serialization formats, such as between XML and CSV.

We will expose a way the user can register their own generator that can create a set of files in a filesystem from the given schema.

To support the linking of generated files, generators may use others to modify their results, further expand them, or create a link to them. This is specifically useful for documentation, as it should contain links and possibly a copy of generated schema.

\subsection{Artifacts}

There is little difference between generated schemas, data transformations, documentation, and other output files. Based on the general schema and provided configuration, if any, the application shall create a set of files that can either be published on the Web or stored in the file system. All generated files will be denoted as \textbf{artifacts} and created by \textbf{generators}.

We will distinguish two types of artifacts. (i) \textbf{Specification artifacts} do not depend on a concrete schema but use the whole specification. Documentation may be an example of a specification artifact because it generates a single document concerning all schemas. Of course, schema-specific documentation is possible, and it purely depends on user requirements. (ii) The \textbf{schema artifacts} are bound to a concrete general schema and are used to generate transformations or schema documents.

\begin{figure}[h!]\centering
  \begin{tikzpicture}
      \node[dataSpecification,align=center] (ds1) at (-3,0) {Data specification};

      \node[generalSchema,align=center] (s11) at (-4.5,-1.5) {General\\schema};
      \node[generalSchema,align=center] (s12) at (-1.5,-1.5) {General\\schema};

      \node[schema,align=center,cascaded] (sa1) at (-7.5,-1.5) {Specification\\artefacts};

      \node[schema,align=center,cascaded] (sa11) at (-4.5,-3) {Schema\\artefacts};
      \node[schema,align=center,cascaded] (sa12) at (-1.5,-3) {Schema\\artefacts};

      \draw[-latex] (ds1) -- (s11);
      \draw[-latex] (ds1) -- (s12);
      \draw[-latex] (ds1) -- (sa1);

      \draw[-latex] (s11) -- (sa11);
      \draw[-latex] (s12) -- (sa12);
  \end{tikzpicture}
  \caption{Schemas, documentation, and other generated files are artifacts. Artifacts are either schema-specific, generated for every schema, or specification-specific for a given data specification.}
\end{figure}

% Necaskeho pozadavek na OR a hierarchii
% https://github.com/mff-uk/dataspecer/issues/95

% Zacnu tim, co vlastne chci a pak introducnu ten or
% neni to uz pak nahodou problem formalni analyzou

\section{Inheritance}

\begin{requirement}
    The tool shall support class inheritance on a general schema level and in generated schemas. That means it shall be possible to design a schema that validates data where both the base and derived classes can be used. The derived class may have additional properties.
    \label{requirement:inheritance}
\end{requirement}

\begin{showcase}
  We will start directly with an example. Suppose that the warehouse also distributes foods in addition to general goods. Food is a type of good, but it may have additional attributes for storage purposes, such as \textit{storing temperature} or \textit{expiration date}. Suppose that we want to design a schema for a JSON list of goods, as seen in \autoref{analysis:inheritance:json-data}. The document is an array of objects, where each object has basic properties such as \textit{name} and \textit{price}. If the object represents food, we want it to have the additional attributes. JSON Schema format is capable of supporting this.

  \begin{figure}[H]\centering
      \begin{Verbatim}[commandchars=\\\{\}]
[
  \{
    "name": "Chair",
    "price": 100,
    "type": "furniture"
  \},
  \{
    "name": "Ice-cream",
    "price": 10,
    "type": "food",
    "expirationDate": "2022-07-21",
    "storingTemperature": "frozen"
  \}
]
      \end{Verbatim}
      \caption{Example of JSON data we want to validate. Based on the type of good, the object may have additional properties.}
      \label{analysis:inheritance:json-data}
  \end{figure}

    Without any additional information, we can only say that if the object contains only one of those additional properties, it is not valid, because there is no such class that has only one of them. Nevertheless, we can add a property that specifies the type (or category) of goods and use this to validate the object.
\end{showcase}

This requirement impacts the application at three different levels. (i) The general schema model must have constructs representing the required problem. This is analyzed mainly in the following chapter. (ii) We must somehow represent the inheritance in the user interface. By this, we mean how to show that the class has a specialization in the "bullet list" representation (see \autoref{analysis/general-schema-representation}). (iii) All generators shall understand them and generate a schema that corresponds to the intended result.

\medskip

An advanced reader may point out that the problem can be generalized by introducing a disjunction to the schema. As we will show in the following text, this assumption is correct. Hovewer, designing schemas only with disjunction is a cumbersome and complicated task for less advanced users. Therefore, we still want to provide the ability to work more efficiently with the inheritance.

In ordinary cases, a more general class shares its properties with all its descendants. This can be seen in the example, where the \textit{Ice-cream} has all properties a \textit{Chair} has. In UML modeling and most programming languages, copying the properties is unnecessary as they are inherited. We want to achieve a similar thing in our schema representation, not polluting the page with redundant information. This, however, restricts the use of inheritance because we can not select the order of properties in the descendant class if the properties are not shown.

To formalize the restriction, a descendant class implicitly has all properties of the parent in the same order, and those properties are before any other properties of the descendant class. This is a limitation for XML and CSV documents as JSON does not depend on the order of properties. However, order generally does not play a huge role in modeling. For advanced use cases, the low-level constructs introduced later can be used.

The rule is applied through the whole chain of inheritance. If there are classes $A$, $B$, and $C$, where $C$ is a descendant of $B$ and $B$ is a descendant of $A$, then $C$ has properties of $A$, properties of $B$, and then its own properties.

In some situations, we might want to omit the parent class from a schema. Let us have a base class $B$ and its two descendants $M$ and $N$. So far, we can model a schema with either $B$, $M$, or $N$, where $B$ provides properties to both $M$ and $N$. $B$ may not represent a real thing per se. It may correspond to an abstract class that serves only as a base for the other classes. In that scenario, we only want to allow $M$ and $N$ to be used.

\medskip

Because we do not want to show inherited properties, the descendant classes must visually belong to their parent to understand that it shares its properties. Hence, we propose that if a class has a specialization in a schema, the specialization will be shown after the properties of the parent. If the base class should be omitted from the schema but has some properties, it will be shown visually differently.

\begin{figure}[h!]\centering
  \begin{Verbatim}[commandchars=\\\{\}]
{\color{purple!60}class \textbf{Good}}
  {\color{blue!60}- attribute \textbf{name}}[1..1]: string
  {\color{blue!60}- attribute \textbf{price}}[1..1]: number
  {\color{purple!60}specialization \textbf{Food}}
    {\color{blue!60}- attribute \textbf{expirationDate}}[1..1]: string
    {\color{blue!60}- attribute \textbf{storingTemperature}}[1..1]: number
    {\color{purple!60}specialization \textbf{Fruit and Vegetables}}
    ...
  {\color{purple!60}specialization \textbf{Drink}}
    ...
\end{Verbatim}
  \caption{Proposition for representing an inheritance in the schema. The row with specialization classes is below the property list and does not belong to it.}
\end{figure}

So far, the internal schema model, although not formally defined, does not support constructs for the purpose of inheritance. We can build on the current proposal of the UI, but this approach is not robust enough. We want to compose the desired results from low-level constructs.

\subsection{Disjunction in schemas}

First, we introduce the concept of a disjunction. The disjunction in a schema context is a set of rules (or subschemas) where exactly one rule must be satisfied. The disjunction serves as the "OR" operator in the schema. We can use it in our inheritance problem to create an association to a disjunction of classes that have a common ancestor. It nicely solves part of the inheritance problem and can be used for other things, as well. For example, the title can be either a string or an object of language and translation pairs.

Both the JSON Schema and the XML Schema support some kind of disjunction. JSON Schema has the {\tt anyOf} keyword, which specifies that the given value must match at least one rule, effectively creating an OR. XSD has {\tt xs:choice} element doing the similar thing.

There are two ways to implement the OR operator in our proposed hierarchical model: on a \textbf{class level} and \textbf{property level}. The former approach allows the OR operator to be placed anywhere where classes can be placed. Either in the root of the schema or in the association. The OR is then a set of classes. The latter approach is more flexible, allowing users to specify the disjunction between tuples of properties.

\begin{figure}[h!]\centering
  \begin{subfigure}[b]{.5\textwidth}
    \centering
    \begin{tikzpicture}[
      type-or/.style={diamond, draw=orange!60, fill=orange!5, very thick},
      type-class/.style={rectangle, draw=red!60, fill=red!5, very thick},
      type-attribute/.style={rectangle, draw=blue!60, fill=blue!5, very thick},
      level 1/.style = {sibling distance = 3.5cm, level distance = 1cm},
      level 2/.style = {sibling distance = 1cm, level distance = 1cm},
      every child/.style={-latex}
    ]
      \node[type-or] (root) {or}
        child {
          node[type-class] {$C^1$}
            child {node {...}}
            child {node[type-attribute] {$A_1$}}
            child {node {...}}
        }
        child {
          node[type-class] {$C^2$}
            child {node {...}}
            child {node[type-attribute] {$A_2$}}
            child {node[type-attribute] {$A_3$}}
            child {node {...}}
        };
      \draw[-latex] (0,1) -- (root);
    \end{tikzpicture}
    \caption{Class-level OR}
    \end{subfigure}%
    \begin{subfigure}[b]{.5\textwidth}
    \centering
    \begin{tikzpicture}[
      type-or/.style={diamond, draw=orange!60, fill=orange!5, very thick},
      type-class/.style={rectangle, draw=red!60, fill=red!5, very thick},
      type-attribute/.style={rectangle, draw=blue!60, fill=blue!5, very thick},
      type-or-group/.style={circle, draw=orange!60, fill=orange!5, very thick},
      level 1/.style = {sibling distance = 1.5cm, level distance = 1.125cm},
      level 2/.style = {sibling distance = 2cm, level distance = .5cm},
      level 3/.style = {sibling distance = 1cm, level distance = 0.75cm},
      every child/.style={-latex}
    ]
      \node[type-class] (root) {$C$}
        child {node {...}}
        child {
          node[type-or] {or}
            child {node[type-or-group]{} child{node[type-attribute] {$A_1$}}}
            child {node[type-or-group]{}
                child {node[type-attribute] {$A_2$}}
                child {node[type-attribute] {$A_3$}}
            }
        }
        child {node {...}};
      \draw[-latex] (0,1) -- (root);
    \end{tikzpicture}
    \caption{Property-level OR}
    \end{subfigure}%
  \caption{Comparison of two models of the semantically same subschemas of class $C$ having either attribute $A_1$, or both $A_2$ and $A_3$.}
\end{figure}

The former model is more common for programmers, as in some languages (such as TypeScript), it is possible to specify a type of property in this particular way as an OR of multiple types. On the other hand, the latter approach is more well known in data modeling, as XSD's {\tt xs:choice} works exactly the same way.

The model using a property-level OR can not use the disjunction in the root of the schema, which is an essential disadvantage as there may be use-cases for those schemas. On the other hand, this model is better suited for the Cartesian product of multiple disjunctions in a single class. Suppose a schema for class $C$ having the following attributes: the first attribute is either $A_11$ or $A_12$, and the second attribute is either $A_21$ or $A_22$.

\begin{figure}[h!]\centering
  \begin{subfigure}[b]{.6\textwidth}
    \centering
    \begin{tikzpicture}[
      type-or/.style={diamond, draw=orange!60, fill=orange!5, very thick},
      type-class/.style={rectangle, draw=red!60, fill=red!5, very thick},
      type-attribute/.style={rectangle, draw=blue!60, fill=blue!5, very thick},
      level 1/.style = {sibling distance = 2.2cm, level distance = 1cm},
      level 2/.style = {sibling distance = 1cm, level distance = 1cm},
      every child/.style={-latex}
    ]
      \node[type-or] (root) {or}
        child {
          node[type-class] {$C^1$}
            child {node[type-attribute] {$A_{11}^1$}}
            child {node[type-attribute] {$A_{21}^1$}}
        }
        child {
          node[type-class] {$C^2$}
            child {node[type-attribute] {$A_{11}^2$}}
            child {node[type-attribute] {$A_{22}^2$}}
        }
        child {
          node[type-class] {$C^3$}
            child {node[type-attribute] {$A_{12}^3$}}
            child {node[type-attribute] {$A_{21}^3$}}
        }
        child {
          node[type-class] {$C^4$}
            child {node[type-attribute] {$A_{12}^4$}}
            child {node[type-attribute] {$A_{22}^4$}}
        }
        ;
      \draw[-latex] (0,1) -- (root);
    \end{tikzpicture}
    \caption{Class-level OR}
    \end{subfigure}%
    \begin{subfigure}[b]{.4\textwidth}
    \centering
    \begin{tikzpicture}[
      type-or/.style={diamond, draw=orange!60, fill=orange!5, very thick},
      type-class/.style={rectangle, draw=red!60, fill=red!5, very thick},
      type-attribute/.style={rectangle, draw=blue!60, fill=blue!5, very thick},
      type-or-group/.style={circle, draw=orange!60, fill=orange!5, very thick},
      level 1/.style = {sibling distance = 2.2cm, level distance = 1.125cm},
      level 2/.style = {sibling distance = 1cm, level distance = .5cm},
      level 3/.style = {level distance = 0.75cm},
      every child/.style={-latex}
    ]
      \node[type-class] (root) {$C$}
        child {
          node[type-or] {or}
            child {node[type-or-group]{} child{node[type-attribute] {$A_{11}$}}}
            child {node[type-or-group]{} child{node[type-attribute] {$A_{12}$}}}
        }
        child {
          node[type-or] {or}
            child {node[type-or-group]{} child{node[type-attribute] {$A_{21}$}}}
            child {node[type-or-group]{} child{node[type-attribute] {$A_{22}$}}}
        };
      \draw[-latex] (0,1) -- (root);
    \end{tikzpicture}
    \caption{Property-level OR}
    \label{fig:cartesian-product:property-or}
    \end{subfigure}%
  \caption{Comparison of two models for the Cartesian product of disjunctions.}
  \label{fig:cartesian-product}
\end{figure}

As seen in \autoref{fig:cartesian-product}, the model with a class-level OR tends to have wider trees for Cartesian products of disjunctions because we have to create (automatically, of course) each combination.

\subsection{Include in schemas}

Before proceeding with the disjunctions, we will solve the rest of the problem with inheriting properties. Each class may explicitly use attributes and associations that belong to a parent class, as stated in the previous requirements. In the inheritance problem, however, we do not want to do that, as those properties are already set on the parent. Therefore, we would need a mechanism to include those properties from the parent class.

The most straightforward way is to implement classical inheritance, as is known from programming languages, between the physical classes in the model. However, this limits us in some schemas where the order of the attributes matters. Therefore, we will use a new construct \textbf{include}, which can "copy" all properties of the given class and insert them in the place where the include is located. Include is hence a part of class properties alongside attributes and classes. The include with the class-level OR can be fully used to implement the desired inheritance.

All classes that participate in the inheritance are internally under the same OR, as only one of those classes is used in the resulting data representation. Each class, except the base class, has an include as a first element in the property list. The include then points to the nearest parent class.

\begin{figure}[h!]\centering
  \centering
      \begin{tikzpicture}[
        level 1/.style = {sibling distance = 5cm, level distance = 1cm},
        level 2/.style = {sibling distance = 1.75cm, level distance = 1cm},
        level 3/.style = {level distance = 0.75cm},
        every child/.style={-latex}
      ]
        \node[typeor] (root) {or}
          child {
              node[typeclass] (good) {Good}
              child { node[typeattribute] {name} }
              child { node[typeattribute] {price} }
          }
          child {
              node[typeclass] {Food}
              child { node[typeincludes] (includes) {includes} }
              child { node[typeattribute] {exp.\\date} }
              child { node[typeattribute] {storing\\temp.} }
          }
        ;

        \draw[-latex] (0,1) -- (root);
         \path[every node/.style={font=\sffamily\small}]
          (includes) edge [-latex,out=225, in=0]  (good);
      \end{tikzpicture}

    \caption{Proposed schema model that handles the inheritance problem with an include and an OR constructs.}
    \label{fig:cartesian-product:include}
  \end{figure}

\bigskip

Having the include construct allows us to overcome the problem of the cartesian product of disjunctions, which is shown in \autoref{fig:cartesian-product}. This is not a proposed solution as we currently do not have a use case where solving this problem is important. Because the include extracts properties of the included subject, we may combine include to OR to a set of classes, which according to the defined logic, would extract properties of one of the included classes.

\begin{figure}[h!]\centering
  \centering
  \begin{tikzpicture}[
    type-or/.style={diamond, draw=orange!60, fill=orange!5, very thick},
    type-class/.style={rectangle, draw=red!60, fill=red!5, very thick},
    type-attribute/.style={rectangle, draw=blue!60, fill=blue!5, very thick},
    type-or-group/.style={circle, draw=orange!60, fill=orange!5, very thick},
    level 1/.style = {sibling distance = 4cm, level distance = 1cm},
    level 2/.style = {sibling distance = 1.5cm, level distance = 1.25cm},
    level 3/.style = {sibling distance = 1.5cm, level distance = 1cm},
    every child/.style={-latex}
    ]
      \node[type-class] (root) {$C$}
        child {
          node[typeincludes] {include}
            child {
                node[type-or]{or}
                child{
                    node[type-class] {$C$}
                        child {node[type-attribute] {$A_{11}$}}
                }
                child{
                    node[type-class] {$C$}
                        child {node[type-attribute] {$A_{12}$}}
                }
            }
        }
        child {
          node[typeincludes] {include}
            child {
                node[type-or]{or}
                child{
                    node[type-class] {$C$}
                        child {node[type-attribute] {$A_{21}$}}
                }
                child{
                    node[type-class] {$C$}
                        child {node[type-attribute] {$A_{22}$}}
                }
            }
        };
      \draw[-latex] (0,1) -- (root);
    \end{tikzpicture}
  \caption{Using include-to-or construction to achieve the same thing as in \autoref{fig:cartesian-product:property-or}. The OR selects one of the two classes, and the include copies the content to the parent.}
\end{figure}

\subsection{Type coherency}\label{sec:type-coherency}

\td{Tady mi to ještě chybí dopsat.}

Although both constructs \textit{OR} and \textit{include} add complexity to the model, it is still possible to ensure basic type safety rules.

The purpose of an \textit{include} is to take all properties of a given class and insert them in the place where \textit{include} is located. Class


There are no restrictions for an \textit{OR} located in the root of the schema tree as there are no restrictions for classes either. We will analyze reference later

\textit{OR} in association, on the other hand, is restricted by the type of the class that is on the association end. We have already stated that only the given class (that the association points to) and its descendants are allowed to be at the association's end. This, however, nicely fits with the \textit{OR} logic as its type can be determined as the nearest common ancestor of all classes in the \textit{OR}.

The last use case for the \textit{OR} was in the include-to-or constructs.

% pak jsem uplne zapomnel na popis toho, jak treba pridat koren, nebo jak pridavat properties a ze i tam je dedicnost
% zamyslet se nad tim, jak bude fungovat lifting a lowering

% potom v te formalni popsat ty dva typy oru, jeden je generalization, jeden specialization

% jeste jak se pridavaji veci jako atributy a ze je mozne pridavat z nadrazenych trid



\section{Future requirements}\label{chapters:future-requirements}

The following requirements in this section are analyzed because they may affect the final model that will be discussed in the next chapter. But due to its complexity, full implementation and analysis will be kept as authors' future work and this thesis covers only the necessity to not introduce a technical debt. The requirements follow the authors' intention of creating a whole ecosystem that supports advanced features of sharing and managing schemas.

\begin{requirement}
  \label{requirement:pim-editing}
  The approach of previous tools for creating the ontology directly in the application is not required, but there should be support for \textit{some} modifications.
\end{requirement}

As stated in \autoref{requirement:ontologies-on-the-web}, the preferred way is to create a complete ontology externally and keep it up to date and valid against the requirements of all involved parties.

However, there may be scenarios where it may be beneficial to change the ontology directly. Some of them are the following:

\begin{enumerate}
  \item The ontology is wrong and does not describe the domain correctly. - \textit{Then, a correct way would be to fix the ontology.}
  \item The ontology describes only a subset of the domain. Either only the core of the domain or the ontology is complete, but only for one domain, whether in another, something may be missing. - \textit{If the desired ontology is strictly a superset of the domain, we can exploit the linked data features to add missing annotations in our own structured data. Then we would use the new ontology.}
  \item The ontology is not granular enough. Some entities can be represented in more detail than they currently are or vice versa. - \textit{We would need to create a copy of affected classes or use an advanced tool if it exists.}
\end{enumerate}

Suppose the example with goods in the delivery company. Although the goods may be identified by EAN (barcode on items), the software team may prefer their own internal identifiers. There may be reasons for not including the identifier in the ontology, as it is too specific for only a software team, for example. This would correspond to the second category from the list above. The missing attribute then may belong to either the original class or the new extended class in the modified ontology. The third category may represent the case where, for example, we need to replace an address with a set of more specific attributes such as \textit{street}, \textit{number}, \textit{city}, \textit{country}, etc.

Although in all scenarios, the preferred way would be to create a new ontology or modify the remaining, it can be too cumbersome and time-consuming, especially if the data modeler wants to try something with an altered ontology, or if a small error needs to be fixed quickly. Therefore, the opportunity of modifying the ontology should be possible.

Allowing such changes must be done carefully as it may interfere with some mechanisms.
\begin{enumerate}
  \item If the original ontology changes, the local overwrites may need to be changed as well; otherwise, they may become invalid. Overwritten data may be removed or moved elsewhere. The evolution mechanism (that is described in more detail in \autoref{requirement:evolution}) hence must work with the overwrites as well.
  \item Moreover, the overwritten data may change, which can lead to two scenarios. Either the user wishes to keep the local version as if nothing happened, or they may want to discard the local version as the new version fixes the issue that caused the modification in the first place.
\end{enumerate}

This issue is too complex, and due to the nature of the requirement, it must be solved directly in the application. We keep the question behind this problem open and focus only on simple modifications, as this will cover most use cases.\footnote{From our specific use case on the Semantic government vocabulary (SGOV), most local changes consist of adding a missing cardinality or fixing labels and descriptions.}

\begin{requirement}
    As there shall be a support for data transformations between different schemas, the data transfomations shall respect various ontology alignments to transform data between different ontologies. Alignments shall be created also during user modification of the ontology, between the modification and the original ontology.
\end{requirement}

\textbf{Alignment} as defined in TODO is a set of relations between entities \textit{usually} from different ontologies. These relations specify the semantic equivalence between the entities and create a mapping that can transform data from one ontology to another.

There are already well-known RDF predicates that can cover basic alignment. For semantically identicall entities, we may use \verb|owl:equivalentClass| or \verb|skos:exactMatch|. More usefull RDF predicate is \verb|rdfs:subClassOf| for more specific classes representing things.

The later one is already included in the previous requirement TODO. Subclasses (i) reuses attributes and associations from its parent class, but also semantically denotes, that (ii) the subclass can also be treated as "the parent class". The second point is an example of a simple ontology alignment.

\begin{showcase}
    % example with address
\end{showcase}

\begin{requirement}
    It shall be possible to perform an evolution of schemas and other artifacts from an ontology. The evolution shall be automatic, if possible, and shall also transform the data that conform to the given schemas and deduce the changes from an ontology that does not support versioning.
    \label{requirement:evolution}
\end{requirement}

Although designing the schemas with the documentation may seem like a one-time job, later management of the schemas is also essential. User requirements may change, resulting in a change in the ontology and underlying schemas. The change may be as simple as adding a new property but can also be more complex, such as splitting classes, moving attributes, or changing their semantics.

As the tool's purpose is to support the whole process of designing the schemas, it shall also provide the possibility to change the schemas in the future easily. We can analyze this requirement on two levels: how to change the schemas and how the change is reflected.

Our current goal is not to create a complex model capable of any change but rather to create a simple, easy-to-maintain solution that can handle most cases. Moreover, for complex changes, it may be cleaner to recreate the schemas from scratch without the need for any evolution mechanism.

\paragraph{Changing the schemas} The source of the change is the ontology, as we are interested only in a top-down (from the ontology to schemas) modeling. Because we want to support all kinds of ontologies, we cannot have additional requirements, such as the history of changes. Therefore, the tool needs to have a mechanism to analyze the ontology in the current state and generate a list of changes.

Having a list of changes and the previously designed schemas, we can perform the evolution. Depending on the context and the user preferences, some changes may be performed automatically. For example, suppose that \textit{name} of the goods is changed to the \textit{title}. This change is simple, and since we are performing the evolution, we probably want the change to be applied as is. On the other hand, some changes may be more complex, where user interaction is necessary.

In any case, the result of the evolution is a new general schema that conforms to the ontology. We can use this general schema to re-generate documentation and schemas for desired languages. In some cases, this may be sufficient, and the work ends here.

\paragraph{Reflecting the changes} Nevertheless, some users may not be satisfied with just a new version of schemas and documentation, as it may be difficult to find out what has changed and how. To painlessly apply the changes in their systems and  to understand the change, they may require:

\begin{enumerate}
    \item \textbf{Data transformations between the old and new schemas} to easily convert the data to a new representation. This may be useful as a temporary workaround to switch to a new format without actually changing the application that uses it. Transformations, of course, can be used to convert all data to the current format if data are stored in it.
    \item \textbf{A document describing what has changed} to easily understand and apply those changes. The document format can be, for example, an HTML file containing the table of renamed attributes, associations, and classes with a textual description of more advanced changes. The purpose of the document can be similar to the documentation and may link other documents and transformation scripts.
\end{enumerate}

Data transformations are de facto already handled by the previous requirements. We will not modify the existing schema during the evolution but rather create a copy. Because both schemas use the same ontology (possibly with alignments), we can generate data transformations between them with RDF as the central format.

Generating documents would probably require a new type of generator that would work on two schemas at once. This, however, is too complex for the current state of the project. Therefore, we will keep this problem for later.


\begin{requirement}
    The tool shall support working with general schemas that are not directly stored in it but may be located in another instance, on the web, or in Solid Pods\footnote{Solid (\url{https://solidproject.org/}) is a specification for storing data in decentralized places called Pods. Users may create Pods in their own servers or use services that provide that option. It is an alternative to services like Facebook or Google that stores data on their servers only.}.
    \label{requirement:schemas-on-the-web}
\end{requirement}

We have already discussed data on the web principle regarding ontology (see \autoref{requirement:ontologies-on-the-web}) as it is preferred to have data published on the web to be easily accessible by anyone. Although this can be achieved in other ways, the great benefit lies in the fact that those data are independent of the tool that created them. Data can be modified and accessed by other tools easily if the tool understands its structure.

In a similar way, we would like to achieve this with all data that represent the state of the schema. Specifically, we mean the structure of the general schema, configuration of all artifacts, other configurations, and helper files. Instead of having an enclosed application that stores all data internally and only provides a way for exporting and importing them, we would like to have ways to read schemas from other sources similarly as they are local and modify them as well if the user is allowed to do so.

This approach allows data modelers to create their own schemas that can be reused by anyone else on the internet. Because the schemas would be hosted by their infrastructure, there is no need for a centralized service that would need to deal with user accounts, GDPR, payments for schema hosting, integration of other tools, etc. Of course, this also means that there would be no repository with search functionality for the schemas.

\medskip

In most cases, storing data externally should not be a problem, as we need to read them from somewhere anyway. If the external storage is inaccessible, the application shall still provide most of its functionality and try to obtain the data later. For example, this may mean that it would not be possible to generate some artifacts, and part of the schema in the UI would not be visible. Because we have introduced data specifications as projects, the problem would only occur when referencing a subschema from a data specification that is stored in the problematic source.

This approach may be problematic if we start changing the schemas. In the current state of the design, schemas can be referenced. If the referenced schema changes (either by evolution or directly by user), the reference may become broken, and referencing schema becomes invalid. In \autoref{subsection:type-coherency} and \ref{sec:type-coherency} we already tickled type coherency.

This, together with the fact that schemas may be modified outside the tool, has major implications as some checks on schemas must be performed continuously and not just during the construction of the schema. Generally, that would mean that schemas may be invalid/broken at any time, and the tool shall still be able to work with them.

\begin{requirement}
    It shall be possible to extend any existing general schema by adding or modifying some of its parts. The extended schema shall remain linked to the original one and allow propagation of changes if the original schema is modified.
    \label{requirement:schema-inheritance}
\end{requirement}

As an example, suppose someone designs and publishes a general schema (not the generated JSON or XML schemas, but the data specification with the general schema itself).

\begin{itemize}
    \item The most common scenario is that we work with data that conforms to the schema as is. For example, the author of the schema publishes the data in one of the formats, and we only need to process them. For this, we only need to generate schemas from the published general schema.
    \item An advanced scenario is that we need to wrap the data and send them elsewhere. Hence we need to create a new schema containing the original one. In this case, the schema reference (see \autoref{analysis/requirement/schema-reference}) is sufficient as we do not modify the content of the payload. This is shown in \autoref{fig:schema-inheritance:json-data-unaltered}.
    \item This requirement addresses a scenario where the payload is somehow modified. For example, we may want to create a proxy that removes personal information from the payload if the user is not logged in. This is depicted in \autoref{fig:schema-inheritance:json-data-censored}. Other examples are to add a timestamp directly to the payload or add additional information to some parts of the data.
\end{itemize}

\begin{figure}[h!]\centering
  \begin{subfigure}{\textwidth}
  \begin{Verbatim}[commandchars=\\\{\}]
{\color{gray!60}\{}
  {\color{gray!60}"name": "John Doe",}
  {\color{gray!60}"role": "customer",}
  {\color{gray!60}"e-mail": "jd@example.com"}
{\color{gray!60}\}}
    \end{Verbatim}
    \caption{JSON data that conforms to the original schema. (the payload)}
  \end{subfigure}


  \begin{subfigure}[b]{.45\textwidth}

    \begin{Verbatim}[commandchars=\\\{\}]
\{
  "recipientPerson": {\color{gray!60}\{}
    {\color{gray!60}"name": "John Doe",}
    {\color{gray!60}"role": "customer",}
    {\color{gray!60}"e-mail": "jd@example.com"}
  {\color{gray!60}\}},
  "message": "Summer sale!"
\}
    \end{Verbatim}
    \caption{JSON data containing the unaltered payload from above.}
    \label{fig:schema-inheritance:json-data-unaltered}
  \end{subfigure}\hfill%
  \begin{subfigure}[b]{.45\textwidth}
    \begin{Verbatim}[commandchars=\\\{\}]
{\color{gray!60}\{}
  {\color{gray!60}"name": "John Doe",}
  {\color{gray!60}"role": "customer",}
  {\color{gray!60}"e-mail":} null
{\color{gray!60}\}}
    \end{Verbatim}
    \caption{JSON data of the payload with censored {\tt e-mail} as it is the personal information.}
    \label{fig:schema-inheritance:json-data-censored}
    \end{subfigure}%
  \caption{Example of the second and third scenario from \autoref{requirement:schema-inheritance}.}
\end{figure}

Similar to reference in schemas (\autoref{analysis/requirement/schema-reference}), it shall be possible to extend any schema from any data specification. Without the need for evolution (\autoref{requirement:evolution}), it is sufficient to simply copy the whole data specification and modify it directly. But in situations where the data depend on other data that conforms to the specification, it is better to have schemas linked to propagate the changes automatically.

As in the previous requirements, we are interested only in minor changes, as for large modifications, it may be impossible to perform evolution, and if so, there would be many possible solutions, which would effectively undermine the whole purpose of the schema extension, which is to not create additional work for the data modeler.

Below we show a sample set of operations for which, under some conditions, it should be simple to implement the evolution. The detailed analysis of the problem is left for future work.

\paragraph{Removal of an entity} If an entity is removed from the derived schema, then any changes to that entity shall simply be ignored. Change of order of properties on the parent class can be performed without a problem simply by applying the new order without the removed property (as the entity must be connected to some class by association). Nevertheless, if the entity is later used somewhere else (for example, in another class by including it), there can be two appropriate actions. Either not include it as it was removed or include it normally as it was meant to be removed from the parent's property list only.

\paragraph{Addition of new property} Creating new entities does not bring any issues as those entities cannot collide with those from the child schema. If the entity is added to a list of properties, it is still possible to change the order in the parent schema as the added property, for example, can keep its absolute position in the list.

\paragraph{Changing the options} Restricting cardinalities, changing titles, and specifying names and descriptions should be possible. If the parent schema changes those values, the tool shall ask the user whether to accept the change or not.

