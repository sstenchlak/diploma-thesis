\chapter{Analysis and formal description of CIM and PIM layers}
\label{chapters:analysis}

\begin{requirement}
    \label{requirement:ontologies-on-the-web}
    As many ontologies are located on the web in formats like OWL, RDFs, UFO and etc.,
    \begin{enumerate*}[label=\textit{(\alph*)}]
        \item the application must support reading them.
        \item The previous approach of creating the ontology directly in the application is not required, but there should be a support for \textit{some} modifications.
      \end{enumerate*}
\end{requirement}

In contrast with the approach introduced in \textit{XCase} and \textit{eXolutio} tools, the process of creating the domain ontology is moved from the application to the extednal tools. The application then only uses those existing ontologies in standart situations.

Before we proceed to formal description, lets focus on impacts of first part of the requirement.

\begin{enumerate}
    \item The ontology may \textbf{not be always avaiable}. This should not forbid us to generate the schemas and make small changes in the schemas if those changes are not directly related to reading the ontology.
    \item The ontology may \textbf{change unexpectedly}. As there are no strict requirements, we must not expect to get the history of changes, yet still be able to perform schema evolution in some semi-automatic way.
    \item The entities in the ontology may \textbf{point to another ontology} according to Linked Open Data principles.
\end{enumerate}

To support \nth{1} and \nth{2} point, the previously introduced five-level framework was modified. We have added new top-most level called CIM (from \textit{Computational Independent Model}). CIM represents the domain ontology. Although there can be more ontologies, with LOD approach we can see them as one. PIM is then used as a copy of CIM and only the necessary entities are copied to PIM. This modification is in accordance with point 1 and gives us base ground for implementing point 2, as we can compare CIM and PIM and try to create a sequence of operations that modifies PIM to be consistent with CIM.

\begin{requirement}
    \label{requirement:multiple-technologies}
    The application shall support multiple serialization technologies (such as JSON Schema, XSD and CSV schema) and it shall be possible to easily add support fot other.
\end{requirement}

Although the requirement \ref{requirement:multiple-technologies} mainly affects the PSM level, it has also impact to the definition of ontology. As we shall support all kinds of serialization technologies, some of them may not use all the information, which forces us to define the ontology and PIM layer in simpliest, most elementary way.

\begin{definition}[PIM] PIM is a triple $S=(S_c, S_a, S_r)$ of sets of classes, attributes and associations, respectively such that:
\begin{itemize}
    \item Attribute $A \in S_a$ belongs to class $C \in S_c$, which is denoted by function $\textrm{class}: S_a \rightarrow S_c$ as $\textrm{class}(A)=S$.
    \item Association $R \in S_r$ is a set of exactly two association ends ${E_1, E_2}$ that are associated with classes similarly as with attributes by $\textrm{class}: E \rightarrow S_c$.
\end{itemize}
\end{definition}

PIM then can be decorated by various semantic and syntactic anotations:

\begin{itemize}
    \item Classes, attributes, associations and association ends may have title and description, or potentially other describing properties that are not directly used in schema generation. However, the title may be used to propose naming of entities' labels in the PSM level.
    \item Attributes and association ends have cardinalities.
    \item Attributes have data types.
\end{itemize}

\begin{definition}[CIM]
CIM $O$ is an ontology database for which function \textit{CIM adapter} $A$ exists, such that $A(O) = S$ is a valid PIM, where every entity $I$ has unique interpretation $interpretation(I)$. Interpretations must be stable and consistent.
\end{definition}

The definition tells us, that CIM is everything that can be translated to PIM and that process is stable over time. If CIM is slightly changed, also the resulting PIM will be changed only slightly. If CIM is stored in LOD format, than the $interpretation$ function returns the URI of the CIM entity, that corresponds to the given PIM entity.

For simplification, in the rest of the thesis we may ommit that CIM "needs to be translated" to PIM and suppose that is already in PIM-like format.

\begin{definition}[consistency]
    We say that \textbf{PIM is consistent with CIM}, if PIM is a subset of CIM.
\end{definition}

From the descriptions above, the inconsistency may happen only if the CIM changes. It is easy to detect it as we can compare entities in PIM with those in CIM. To support the \nth{2} point from requirement \ref{requirement:ontologies-on-the-web}, we would need to create a set of atomic operations on the PIM level that make the PIM consistent again. These operations can be propagated up to schemas to apply the change from the ontology.

\section{User modifications on the PIM level}

As stated in the requirement \ref{requirement:ontologies-on-the-web}, the usual scenario for data modelling consists of reading an ontology from the web as a CIM. This section analyzes the second part of the requirement, whether there exists scenarios, where modifying the ontology in the tool may be beneficial than modification of the ontology itself.


We can classify the reasons for modifying the ontology as follows:

\begin{enumerate}
    \item The ontology is wrong and does not describe the domain correctly. - \textit{Then the most correct way would be to fix the ontology.}
    \item The ontology describes only a subset of the domain. Either only the core of the domain, or the ontology is complete, but only for one domain, whether in another, something may missing. - \textit{If the desired ontology is strictly a superset of the domain, we can exploit the features of linked data to add missing anotations in our own structured data. Then, we would use the new ontology.}
    \item The ontology is not granular enough. Some entities can be represented in more details than currently are. - \textit{We would need to create a copy of affected classes or use advanced tool, if exists.}
\end{enumerate}

Suppose the example with goods in the delivery company. Although the goods may be identified by EAN (barcode on items), the software team may prefer the other, iternal identifiers. There can be reasons for not including the identifier into the ontology, as it is too internal or specific for only a software team. That would correspond to the second category. The third category may represents the case, when we need to replace an address with a set of more specific attributes such as \textit{street}, \textit{number}, \textit{city}, \textit{country}, etc.

Although in all the scenarios, the preffered way would be to create a new ontology, it can be too cumbersome and time consuming. Therefore, it should be possible to somehow allow modifying the PIM. PIM, that has been modified will be denoted as "user PIM" or \textbf{UPIM}.

\bigskip

In context of other requirements and the framework used, this would also mean that:
\begin{itemize}
    \item PIM that is not consistent with CIM and we do not intent to make it consistent can be "marked" as UPIM. This would stop proposing the user to evolute the schemas according to the newest CIM.
    \item If CIM changes, the semi-automatic update can respect UPIM and ask the user for changes, that can not be performed automatically.
\end{itemize}

We propose to add UPIM to the framework as a new level between PIM and PSM. The purpose of UPIM will be modyfying the ontology stored in PIM by adding, or overwriting the entities.

Construct used in UPIM must be chosen wisely as we do not wont to create too complex ontologies that will be hard to maintain both by user and the evolution mechanism. Yet, it is prefered to give user more freedom than restricted, but more formal and stable model.

We allow all constructs from PIM as it shoud be definitelly possible to move everything from PIM to UPIM.